\chapter{Annexe}

\subsection*{main.cpp}
\begin{minted}[
    breaklines=true,
    frame=lines,
    linenos
    ]{c++}
    #include <iostream>
    #include <stdlib.h>
    #include<time.h>
    #include <bits/stdc++.h>
    
    #include "AlgoHanoi.hpp"
    
    using namespace std;
    
    int main()
    {
        srand(time(0));
        cout <<"\t\t\tBienvenu dans la partie 2 du projet de Conception, analyse et complexite des algorithmes"<<endl;
        cout <<"Veuille choisir l'une des options suivante :"<<endl;
        cout <<"1) Algorithme des d'Hanoi recursif"<<endl;
        cout <<"2) Algorithme des d'Hanoi recursif avec affichage"<<endl;
        cout <<"3) Algorithme des d'Hanoi iteratif"<<endl;
        cout <<"4) Algorithme des d'Hanoi iteratif avec affichage"<<endl;
        cout <<"5) Solution aleatoire"<<endl;
    
        int ch, NbrDisque;
        Pilier TourHanoi[3];
        cin >> ch;
        chrono::time_point<chrono::steady_clock> start, stop;
        chrono::duration<double, nano> duration;
        switch(ch)
        {
            case 1:
                cout << "Choisissez le nombre de disque" << endl;
                cin >> NbrDisque;
                NDisque = NbrDisque;
                init(TourHanoi, NbrDisque);
                start = chrono::steady_clock::now();
                HanoiRecursif(NbrDisque, 0, 2, TourHanoi);
                stop = chrono::steady_clock::now();
                duration = stop - start;
                cout << "Tour d'hanoi resolu en " << deplacement << " deplacements" <<endl;
                cout << "la fonction de resolution de la tour d'hanoi aura prit " << duration.count()
                     << "ns afin de terminer son execution" <<endl;
                break;
    
            case 2:
                cout << "Choisissez le nombre de disque" << endl;
                cin >> NbrDisque;
                NDisque = NbrDisque;
                init(TourHanoi, NbrDisque);
                AfficheLesTours(TourHanoi, NbrDisque);
                Sleep(500);
                HanoiRecursifAffichage(NbrDisque, 0, 2, TourHanoi);
                cout << "Tour d'hanoi resolu en " << deplacement << " deplacements" <<endl;
                break;
    
            case 3:
                cout << "Choisissez le nombre de disque" << endl;
                cin >> NbrDisque;
                NDisque = NbrDisque;
                init(TourHanoi, NbrDisque);
                start = chrono::steady_clock::now();
                HanoiIteratif(NbrDisque, 0, 2, TourHanoi);
                stop = chrono::steady_clock::now();
                duration = stop - start;
                cout << "Tour d'hanoi resolu en " << deplacement << " deplacements" <<endl;
                cout << "la fonction de resolution de la tour d'hanoi aura prit " << duration.count()
                     << "ns afin de terminer son execution" <<endl;
                break;
    
            case 4:
                cout << "Choisissez le nombre de disque" << endl;
                cin >> NbrDisque;
                NDisque = NbrDisque;
                init(TourHanoi, NbrDisque);
                AfficheLesTours(TourHanoi, NbrDisque);
                Sleep(500);
                HanoiIteratifAffichage(NbrDisque, 0, 2, TourHanoi);
                cout << "Tour d'hanoi resolu en " << deplacement << " deplacements" <<endl;
                break;
    
            case 5:
                cout << "Choisissez le nombre de disque" << endl;
                cin >> NbrDisque;
                NDisque = NbrDisque;
                init(TourHanoi, NbrDisque);
                start = chrono::steady_clock::now();
                Aleatoire(NbrDisque, TourHanoi);
                if (Verification(NbrDisque, TourHanoi))
                    cout << "Solution valide" << endl;
                else
                    cout << "Solution invalide" << endl;
                stop = chrono::steady_clock::now();
                duration = stop - start;
                cout << "la fonction de generation de solution de tour d'hanoi et de verification aura prit " << duration.count()
                     << "ns afin de terminer son execution" <<endl;
                break;
    
            default:
                cout << "erreur "<<endl;
        }
    }
\end{minted}

\subsection*{Affichage.hpp}
\begin{minted}[
    breaklines=true,
    frame=lines,
    linenos
    ]{c++}
    #ifdef _WIN32
    #include <Windows.h>
    #else
    #include <unistd.h>
    #endif
    
    #include <iostream>
    #include <cstdlib>
    
    #include "structure.hpp"
    
    using namespace std;
    
    //affiche un char n fois
    void affiche(char c, int n)
    {
      for (int i = 0; i < n; ++i)
        cout << c;
    }
    
    //affiche une tour
    void AfficherTour(int d, int n)
    {
      //pas disque affihe la poutre du pilier
      if (d == 0) {
        affiche(' ', n-1);
        cout << '|';
        affiche(' ', n);
      }
      //affichage du pilier selon sa taille
      else {
        affiche(' ', n-d);
        affiche('-', 2*d-1);
        affiche(' ', n-d+1);
      }
    }
    
    //affiche des 3 tours
    void AfficheLesTours(Pilier TH[], int n)
    {
      for (int i = 0; i < n; ++i) {
        AfficherTour(TH[0].Pilier[i], n);
        AfficherTour(TH[1].Pilier[i], n);
        AfficherTour(TH[2].Pilier[i], n);
        cout << endl;
      }
    
      affiche('#', 6*n-1);
      cout << endl << endl;
      Sleep(500);
    }
\end{minted}

\subsection*{AlgoHanoi.hpp}
\begin{minted}[
    breaklines=true,
    frame=lines,
    linenos
    ]{c++}
    #include <iostream>
    #include <stdlib.h>
    #include <math.h>
    #include <vector>
    #include <time.h>
    
    
    #include "Affiichage.hpp"
    
    using namespace std;
    
    struct Dep
    {
        int o, d;
    };
    
    //variable globale afin de sauvegarder le nombre de deplacements et de disques
    int deplacement = 0, NDisque = 0;
    
    //resolution recursif des tours d'hanoi
    void HanoiRecursif(int n, int origine,int destination, Pilier TH[])
    {
        if (n != 0)
        {
            //on determine le pilier intermediaire
            int auxiliaire = autre(origine, destination);
            //premier appel recursif ou on deplace le disque de son pilier actuel au pilier intermediaire
            HanoiRecursif(n-1, origine, auxiliaire, TH);
            //on effectue le deplacement
            deplacement++;
            deplacer(TH[origine], TH[destination],NDisque);
            //second appel recursif ou on deplace le disque du pilier intermediaire au pilier destiination
            HanoiRecursif(n-1, auxiliaire, destination, TH);
        }
    }
    
    //resolution recursif des tours d'hanoi + affichage
    void HanoiRecursifAffichage(int n, int origine,int destination, Pilier TH[])
    {
        if (n != 0)
        {
            //on determine le pilier intermediaire
            int auxiliaire = autre(origine, destination);
            //premier appel recursif ou on deplace le disque de son pilier actuel au pilier intermediaire
            HanoiRecursifAffichage(n-1, origine, auxiliaire, TH);
            //on effectue le deplacement
            deplacement++;
            deplacer(TH[origine], TH[destination],NDisque);
            //affichage intermediaire des tours
            AfficheLesTours(TH, NDisque);
            //second appel recursif ou on deplace le disque du pilier intermediaire au pilier destiination
            HanoiRecursifAffichage(n-1, auxiliaire, destination, TH);
        }
    }
    
    /*
    Afin de resoudre ce problème iterativement on applique les regles suivantes :
    - déplacer le plus petit disque d'un emplacement à l'emplacement suivant (de TH[0] vers TH[1], de TH[1] vers TH[2], de TH[2] vers TH[0]) ;
    - déplacer un autre disque ;
    */
    void HanoiIteratif(int n, int origine,int destination, Pilier TH[])
    { 
        //on initialise le pilier d'origine ainsi que le pilier destination du premier deplacement
        int op = 0, dp = 1;
        while(TH[2].Pilier[0] == 0)
        {
            //on effectue le deplacement du plus petit disque
            deplacer(TH[op], TH[dp],NDisque);
            deplacement++;
            //on determine les 2 piliers n'occupant pas le plus petit disque 
            Chemin ep = pilierInter(dp);
            //si le pilier destination n'est pas deja remplie 
            if(TH[2].Pilier[0] == 0)
            {
                //on recupere les disques se trouvant au sommet des 2 piliers 
                int a = sommet(TH[ep.o],NDisque), b = sommet(TH[ep.d],NDisque);
    
                /*ensuite on compare lequel des 2 piliers possedent le disque le plus petit ce dernier sera par la suite
                deplacer vers le second, dans le cas ou un des piliers est vide on deplace le disque de l'autre pilier
                automatique*/
                if(a== NDisque)
                {
                    deplacer(TH[ep.d], TH[ep.o],NDisque);
                    deplacement++;
                }
                else if (b == NDisque)
                {
                    deplacer(TH[ep.o], TH[ep.d],NDisque);
                    deplacement++;
                }
                else if(TH[ep.o].Pilier[a] > TH[ep.d].Pilier[b])
                {
                    deplacer(TH[ep.d], TH[ep.o],NDisque);
                    deplacement++;
                }
                else
                {
                    deplacer(TH[ep.o], TH[ep.d],NDisque);
                    deplacement++;
                }
            }
            //on determine le pilier actuel du plus petit disque ainsi que sa destination
            op = op == 2 ? 0 : op+1; 
            dp = dp == 2 ? 0 : dp+1;
        }
    }
    
    //meme chose que la procedure precedente avec ajout des procedure d'affichage
    void HanoiIteratifAffichage(int n, int origine,int destination, Pilier TH[])
    { 
        int op = 0, dp = 1;
        while(TH[2].Pilier[0] == 0)
        {
        deplacer(TH[op], TH[dp],NDisque);
        AfficheLesTours(TH, NDisque);
        deplacement++;
        Chemin ep = pilierInter(dp);
        if(TH[2].Pilier[0] == 0)
        {
            int a = sommet(TH[ep.o],NDisque), b = sommet(TH[ep.d],NDisque);
    
            if(a == NDisque)
            {
            deplacer(TH[ep.d], TH[ep.o],NDisque);
            deplacement++;
            }
            else if (b == NDisque)
            {
            deplacer(TH[ep.o], TH[ep.d],NDisque);
            deplacement++;
            }
            else if(TH[ep.o].Pilier[a] > TH[ep.d].Pilier[b])
            {
            deplacer(TH[ep.d], TH[ep.o],NDisque);
            deplacement++;
            }
            else
            {
            deplacer(TH[ep.o], TH[ep.d],NDisque);
            deplacement++;
            }
            AfficheLesTours(TH, NDisque);
        }
        op = op == 2 ? 0 : op+1; 
        dp = dp == 2 ? 0 : dp+1;
        }
    }
    
    //fonction generant des deplacement aleatoire
    vector<Dep> Aleatoire(int n, Pilier TH[])
    {
        // le nombre de deplacements a generer est 2^n - 1
        vector<Dep> solution;
        srand(time(NULL));
        Dep dep;
        int top_o, top_d;
        bool dep_autorise;
        for (int i = 0; i < pow(2, n) - 1; ++i)
        {
            dep_autorise = false;
            //on genere un deplacement aleatoire valide
            while (!dep_autorise)
            {
                dep.o = rand() % 3;
                if ((top_o = sommet(TH[dep.o], n)) == n) // origine vide
                    continue;
                dep.d = rand() % 3;
                if (dep.d == dep.o) // destination meme que origine
                    continue;
                top_d = sommet(TH[dep.d], n);
                if (top_d == n || TH[dep.o].Pilier[top_o] < TH[dep.d].Pilier[top_d])
                    dep_autorise = true;
            }
            //on l'ajoute a la liste de deplacement
            solution.push_back(dep);
    
            cout << "on deplace le disque de taille "<< TH[dep.o].Pilier[sommet(TH[dep.o], n)] 
                 <<" du pilier "<< dep.o <<" au pilier "<< dep.d << endl;
    
            deplacer(TH[dep.o], TH[dep.d], n); 
        }
        cout << "la solution contient " << solution.size() << " deplacements" << endl;
    
        return solution;
    }
    
    //verifiation de la solution aleatoire
    bool Verification(int n, Pilier TH[])
    {
        for (int i = 0; i < n; ++i)
        {
            //si le pilier final ne contient pas tout les disque dans le bon ordre alors on retourne false
            if (TH[2].Pilier[i] != i + 1)
                return false;
        }
        return true;
    }
\end{minted}

\subsection*{structure.hpp}
\begin{minted}[
    breaklines=true,
    frame=lines,
    linenos
    ]{c++}
    #include <iostream>

    using namespace std;
    
    struct Pilier
    {
      int *Pilier;
    };
    
    struct Chemin
    {
        int o,d;
    };
    
    //initalisation du pilier
    void init(Pilier TH[], int n)
    {
      //creation des 3 piliers
      for (int i = 0; i < 3; ++i)
        TH[i].Pilier = (int *)malloc(n * sizeof(int));
      
      // chaque pilier est representer par sa taille le plus petit est representer par l'indice 1 et le plus grand par l'indice n
      // on represente l'abscence de pilier par l'indice 0 
      for (int i = 0; i < n; ++i)
      {
        TH[0].Pilier[i] = i+1; // premier pilier contient l'integralite des disque dans l'odre de leur taille variant de 1 a n
        TH[1].Pilier[i] = 0; // pilier vide
        TH[2].Pilier[i] = 0; // pilier vide
      }
    }
    
    //on retourne le disque se trouvant au sommet du pilier
    int sommet(Pilier p, int n)
    {
      int top;
      //on parcourt le pilier jusqu'a atteindre le premier disque du tableau 
      for (top = 0; (top < n) && (p.Pilier[top] == 0); ++top);
        return top;
    }
    
    //on deplace un disque d'un pilier a un autre
    void deplacer(Pilier &origine, Pilier &destination, int n)
    {
      //on recupere le disque a deplacer
      int top1 = sommet(origine, n);
    
      if (top1 < n)
      {
        /*on verifie que le disque a deplacer n'est pas plus grand que le disque se rouvant au sommet du pilier destination
        et que le pilier destination n'est pas autrement on retourne une erreur*/
        int top2 = sommet(destination,n);
        if ((top2 < n) && (destination.Pilier[top2] < origine.Pilier[top1])) 
        {
          cerr << "ERREUR : on ne peut pas deplacer une disque de taille "
               << origine.Pilier[top1] << " sur un disque de taille "
               << destination.Pilier[top2] << " !" << endl;
          return;
        }
    
        //on ajoute le disque au pilier destination et on l'enleve du pilier origine
        destination.Pilier[top2-1] = origine.Pilier[top1];
        origine.Pilier[top1] = 0;
      }
    }
    
    //defini le pilier intermediaire
    int autre(int p1, int p2)
    {
      return 3 - p1 -p2;
    }
    
    //retoune les 2 piliers ne contenant pas le plus petit disque
    Chemin pilierInter(int a)
    {
      Chemin ep;
      //la variable 'a' contient la position pilier contenant le plus petit disque on retourne par consequent les 2 autres piliers
      switch(a)
      {
        case 1 :
          ep.o = 0;
          ep.d = 2;
          break;
        case 0 :
          ep.o = 1;
          ep.d = 2;
          break;
        case 2 :
          ep.o = 0;
          ep.d = 1;
          break;
      }
    
      return ep;
    }
\end{minted}