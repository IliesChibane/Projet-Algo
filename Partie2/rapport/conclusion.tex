\chapter{Conclusion générale}

\section{Conclusion}
Dans ce rapport, nous explorons comment le jeu d'enfant classique de la tour de Hanoï peut être utilisé à un niveau élaboré ou élémentaire. Il peut être exploité pour la vulgarisation des concepts mathématiques et pousse le joueur débutant à la réflexion en résolvant le jeu avec 3 ou 4 disques, tandis que les  plus avancés peuvent se poser plusieurs questions sur les liens de la structure avec les systèmes de numération, la modification des règles et d'autres questions qui sont encore l'objet des recherches d'aujourd'hui.\\
Nous avons commencé l'exploration en présentant le problème, son origine et sa définition formelle pour trouver la solution. \\
Dans le deuxième chapitre, nous avons entamer le processus de la résolution en modélisant cette dernière. Nous avons ensuite détaillé les algorithmes de solution et de vérification nécessaires ainsi que le calcul de leurs complexités. Finalement, nous avons illustrer une instance du problème avec la solution générée.  \\
Nous avons fini notre étude avec des expériences pour calculer les complexités théroriques (temporelle et spatiale) de nos algorithmes et les comparer aux résultats expérimentaux obtenus sur plusieurs essais.\\
L'expérimentation avec ce jeu offre à chacun l'occasion d'apprécier les ressorts du jeu, tels que ses aspects algorithmiques, les structures de données, les mathématiques appliquées, etc.\\
Par ailleurs, le jeu de la tour de Hanoï est un cas idéal pour illustrer la puissance
de la notion d’algorithme récursif et donne un exemple du calcul de factorielle $n$. En effet, son efficacité est impressionnante pour seulement quelques lignes de code. Contrairement à ce dernier type d'algorithme, un algorithme itératif de résolution de la tour de Hanoï est en général beaucoup moins facilement trouvé et est toujours plus long.